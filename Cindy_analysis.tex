\documentclass[]{article}
\usepackage{lmodern}
\usepackage{amssymb,amsmath}
\usepackage{ifxetex,ifluatex}
\usepackage{fixltx2e} % provides \textsubscript
\ifnum 0\ifxetex 1\fi\ifluatex 1\fi=0 % if pdftex
  \usepackage[T1]{fontenc}
  \usepackage[utf8]{inputenc}
\else % if luatex or xelatex
  \ifxetex
    \usepackage{mathspec}
  \else
    \usepackage{fontspec}
  \fi
  \defaultfontfeatures{Ligatures=TeX,Scale=MatchLowercase}
\fi
% use upquote if available, for straight quotes in verbatim environments
\IfFileExists{upquote.sty}{\usepackage{upquote}}{}
% use microtype if available
\IfFileExists{microtype.sty}{%
\usepackage{microtype}
\UseMicrotypeSet[protrusion]{basicmath} % disable protrusion for tt fonts
}{}
\usepackage[margin=1in]{geometry}
\usepackage{hyperref}
\hypersetup{unicode=true,
            pdftitle={Untitled},
            pdfborder={0 0 0},
            breaklinks=true}
\urlstyle{same}  % don't use monospace font for urls
\usepackage{color}
\usepackage{fancyvrb}
\newcommand{\VerbBar}{|}
\newcommand{\VERB}{\Verb[commandchars=\\\{\}]}
\DefineVerbatimEnvironment{Highlighting}{Verbatim}{commandchars=\\\{\}}
% Add ',fontsize=\small' for more characters per line
\usepackage{framed}
\definecolor{shadecolor}{RGB}{248,248,248}
\newenvironment{Shaded}{\begin{snugshade}}{\end{snugshade}}
\newcommand{\KeywordTok}[1]{\textcolor[rgb]{0.13,0.29,0.53}{\textbf{#1}}}
\newcommand{\DataTypeTok}[1]{\textcolor[rgb]{0.13,0.29,0.53}{#1}}
\newcommand{\DecValTok}[1]{\textcolor[rgb]{0.00,0.00,0.81}{#1}}
\newcommand{\BaseNTok}[1]{\textcolor[rgb]{0.00,0.00,0.81}{#1}}
\newcommand{\FloatTok}[1]{\textcolor[rgb]{0.00,0.00,0.81}{#1}}
\newcommand{\ConstantTok}[1]{\textcolor[rgb]{0.00,0.00,0.00}{#1}}
\newcommand{\CharTok}[1]{\textcolor[rgb]{0.31,0.60,0.02}{#1}}
\newcommand{\SpecialCharTok}[1]{\textcolor[rgb]{0.00,0.00,0.00}{#1}}
\newcommand{\StringTok}[1]{\textcolor[rgb]{0.31,0.60,0.02}{#1}}
\newcommand{\VerbatimStringTok}[1]{\textcolor[rgb]{0.31,0.60,0.02}{#1}}
\newcommand{\SpecialStringTok}[1]{\textcolor[rgb]{0.31,0.60,0.02}{#1}}
\newcommand{\ImportTok}[1]{#1}
\newcommand{\CommentTok}[1]{\textcolor[rgb]{0.56,0.35,0.01}{\textit{#1}}}
\newcommand{\DocumentationTok}[1]{\textcolor[rgb]{0.56,0.35,0.01}{\textbf{\textit{#1}}}}
\newcommand{\AnnotationTok}[1]{\textcolor[rgb]{0.56,0.35,0.01}{\textbf{\textit{#1}}}}
\newcommand{\CommentVarTok}[1]{\textcolor[rgb]{0.56,0.35,0.01}{\textbf{\textit{#1}}}}
\newcommand{\OtherTok}[1]{\textcolor[rgb]{0.56,0.35,0.01}{#1}}
\newcommand{\FunctionTok}[1]{\textcolor[rgb]{0.00,0.00,0.00}{#1}}
\newcommand{\VariableTok}[1]{\textcolor[rgb]{0.00,0.00,0.00}{#1}}
\newcommand{\ControlFlowTok}[1]{\textcolor[rgb]{0.13,0.29,0.53}{\textbf{#1}}}
\newcommand{\OperatorTok}[1]{\textcolor[rgb]{0.81,0.36,0.00}{\textbf{#1}}}
\newcommand{\BuiltInTok}[1]{#1}
\newcommand{\ExtensionTok}[1]{#1}
\newcommand{\PreprocessorTok}[1]{\textcolor[rgb]{0.56,0.35,0.01}{\textit{#1}}}
\newcommand{\AttributeTok}[1]{\textcolor[rgb]{0.77,0.63,0.00}{#1}}
\newcommand{\RegionMarkerTok}[1]{#1}
\newcommand{\InformationTok}[1]{\textcolor[rgb]{0.56,0.35,0.01}{\textbf{\textit{#1}}}}
\newcommand{\WarningTok}[1]{\textcolor[rgb]{0.56,0.35,0.01}{\textbf{\textit{#1}}}}
\newcommand{\AlertTok}[1]{\textcolor[rgb]{0.94,0.16,0.16}{#1}}
\newcommand{\ErrorTok}[1]{\textcolor[rgb]{0.64,0.00,0.00}{\textbf{#1}}}
\newcommand{\NormalTok}[1]{#1}
\usepackage{graphicx,grffile}
\makeatletter
\def\maxwidth{\ifdim\Gin@nat@width>\linewidth\linewidth\else\Gin@nat@width\fi}
\def\maxheight{\ifdim\Gin@nat@height>\textheight\textheight\else\Gin@nat@height\fi}
\makeatother
% Scale images if necessary, so that they will not overflow the page
% margins by default, and it is still possible to overwrite the defaults
% using explicit options in \includegraphics[width, height, ...]{}
\setkeys{Gin}{width=\maxwidth,height=\maxheight,keepaspectratio}
\IfFileExists{parskip.sty}{%
\usepackage{parskip}
}{% else
\setlength{\parindent}{0pt}
\setlength{\parskip}{6pt plus 2pt minus 1pt}
}
\setlength{\emergencystretch}{3em}  % prevent overfull lines
\providecommand{\tightlist}{%
  \setlength{\itemsep}{0pt}\setlength{\parskip}{0pt}}
\setcounter{secnumdepth}{0}
% Redefines (sub)paragraphs to behave more like sections
\ifx\paragraph\undefined\else
\let\oldparagraph\paragraph
\renewcommand{\paragraph}[1]{\oldparagraph{#1}\mbox{}}
\fi
\ifx\subparagraph\undefined\else
\let\oldsubparagraph\subparagraph
\renewcommand{\subparagraph}[1]{\oldsubparagraph{#1}\mbox{}}
\fi

%%% Use protect on footnotes to avoid problems with footnotes in titles
\let\rmarkdownfootnote\footnote%
\def\footnote{\protect\rmarkdownfootnote}

%%% Change title format to be more compact
\usepackage{titling}

% Create subtitle command for use in maketitle
\newcommand{\subtitle}[1]{
  \posttitle{
    \begin{center}\large#1\end{center}
    }
}

\setlength{\droptitle}{-2em}
  \title{Untitled}
  \pretitle{\vspace{\droptitle}\centering\huge}
  \posttitle{\par}
  \author{}
  \preauthor{}\postauthor{}
  \date{}
  \predate{}\postdate{}


\begin{document}
\maketitle

\begin{Shaded}
\begin{Highlighting}[]
\KeywordTok{rm}\NormalTok{(}\DataTypeTok{list=}\KeywordTok{ls}\NormalTok{())}
\CommentTok{# lost lColor code, for now using dput}
\NormalTok{lColor =}\StringTok{ }\KeywordTok{structure}\NormalTok{(}\KeywordTok{list}\NormalTok{(}\DataTypeTok{black =} \StringTok{"0_0_0"}\NormalTok{, }\DataTypeTok{white =} \StringTok{"255_255_255"}\NormalTok{, }\DataTypeTok{red =} \StringTok{"255_0_0"}\NormalTok{, }
    \DataTypeTok{green =} \StringTok{"0_255_0"}\NormalTok{, }\DataTypeTok{blue =} \StringTok{"0_0_255"}\NormalTok{), }\DataTypeTok{.Names =} \KeywordTok{c}\NormalTok{(}\StringTok{"black"}\NormalTok{, }
\StringTok{"white"}\NormalTok{, }\StringTok{"red"}\NormalTok{, }\StringTok{"green"}\NormalTok{, }\StringTok{"blue"}\NormalTok{))}
\NormalTok{lDf_colors =}\StringTok{ }\KeywordTok{list}\NormalTok{()}
\NormalTok{iou_threshold =}\StringTok{ }\FloatTok{0.2}
\NormalTok{sd_threshold =}\StringTok{ }\DecValTok{3}
\end{Highlighting}
\end{Shaded}

\begin{Shaded}
\begin{Highlighting}[]
\NormalTok{plot_results =}\StringTok{ }\ControlFlowTok{function}\NormalTok{(my_file_name, my_iv, my_net, my_ds)\{}
\NormalTok{  df =}\StringTok{ }\KeywordTok{read.csv}\NormalTok{(my_file_name)}
  \CommentTok{# iou_threshold = 0.2}
  \ControlFlowTok{if}\NormalTok{(my_iv }\OperatorTok{==}\StringTok{ 'back_distractor_target'}\NormalTok{)\{}
\NormalTok{    df =}\StringTok{ }\KeywordTok{read.csv}\NormalTok{(my_file_name)}
\NormalTok{    df01 =}\StringTok{ }\NormalTok{df }\OperatorTok
\StringTok{      }\KeywordTok{select}\NormalTok{(}\KeywordTok{contains}\NormalTok{(}\StringTok{'color'}\NormalTok{)) }\CommentTok{#back_distractor_target}
\NormalTok{    identities =}\StringTok{ }\KeywordTok{c}\NormalTok{(}\StringTok{'back'}\NormalTok{, }\StringTok{'distractor'}\NormalTok{, }\StringTok{'target'}\NormalTok{)}
\NormalTok{    df02 =}\StringTok{ }\NormalTok{df01}
    \ControlFlowTok{for}\NormalTok{(iI }\ControlFlowTok{in}\NormalTok{ identities)\{}
\NormalTok{      df02 =}\StringTok{ }\KeywordTok{unite_}\NormalTok{(df02, iI, }\KeywordTok{colnames}\NormalTok{(}\KeywordTok{select}\NormalTok{(df02, }\KeywordTok{contains}\NormalTok{(iI))))}
\NormalTok{    \}}
\NormalTok{    df03 =}\StringTok{ }\NormalTok{df02}
    \ControlFlowTok{for}\NormalTok{(iC }\ControlFlowTok{in} \KeywordTok{names}\NormalTok{(lColor))\{}
\NormalTok{      df03 =}\StringTok{ }\KeywordTok{data.frame}\NormalTok{(}\KeywordTok{lapply}\NormalTok{(df03, }\ControlFlowTok{function}\NormalTok{(x)\{}
      \KeywordTok{str_replace}\NormalTok{(x, lColor[[iC]], iC)}
\NormalTok{      \}))}
\NormalTok{    \}}
\NormalTok{    df04 =}\StringTok{ }\KeywordTok{unite_}\NormalTok{(df03, my_iv, }\KeywordTok{colnames}\NormalTok{(df03))}
\NormalTok{    df =}\StringTok{ }\KeywordTok{cbind}\NormalTok{(df03, df04, df) }
\NormalTok{  \}}
\NormalTok{  df1 =}\StringTok{ }\NormalTok{df }\OperatorTok
\StringTok{    }\KeywordTok{select}\NormalTok{(}\KeywordTok{starts_with}\NormalTok{(}\StringTok{"iou"}\NormalTok{)) }\OperatorTok{>}\StringTok{ }\NormalTok{iou_threshold}
\NormalTok{  df2 =}\StringTok{ }\KeywordTok{data.frame}\NormalTok{(df1)}\OperatorTok{*}\DecValTok{100}
\NormalTok{  df3 =}\StringTok{ }\NormalTok{df2 }\OperatorTok
\StringTok{    }\KeywordTok{mutate}\NormalTok{(}\DataTypeTok{iv =}\NormalTok{ df[[my_iv]]) }\OperatorTok
\StringTok{    }\KeywordTok{group_by}\NormalTok{(iv) }\OperatorTok
\StringTok{    }\KeywordTok{summarise_all}\NormalTok{(}\KeywordTok{funs}\NormalTok{(}\StringTok{'mean'}\NormalTok{ =}\StringTok{ }\NormalTok{mean))}
  \CommentTok{# df4 = data.frame(t(df3))}
  \CommentTok{# df5 = df4[-1, ]}
  \CommentTok{# colnames(df5) = df4[1,]}
  \CommentTok{# df5$layer = 1:dim(df1)[2]}
  \CommentTok{# df6 = melt(df5, id='layer')}
\NormalTok{  df4 =}\StringTok{ }\KeywordTok{data.frame}\NormalTok{(}\KeywordTok{t}\NormalTok{(df3[, }\OperatorTok{-}\DecValTok{1}\NormalTok{]))}
  \KeywordTok{colnames}\NormalTok{(df4) =}\StringTok{ }\KeywordTok{sapply}\NormalTok{(df3[, }\DecValTok{1}\NormalTok{], as.character)}
\NormalTok{  df4}\OperatorTok{$}\NormalTok{layer =}\StringTok{ }\DecValTok{1}\OperatorTok{:}\KeywordTok{dim}\NormalTok{(df1)[}\DecValTok{2}\NormalTok{]}
\NormalTok{  df6 =}\StringTok{ }\KeywordTok{melt}\NormalTok{(df4, }\DataTypeTok{id=}\StringTok{'layer'}\NormalTok{)}
\NormalTok{  p =}\StringTok{ }\KeywordTok{ggplot}\NormalTok{(df6, }\KeywordTok{aes}\NormalTok{(layer, value, }\DataTypeTok{color=}\NormalTok{variable)) }\OperatorTok{+}
\StringTok{    }\KeywordTok{geom_line}\NormalTok{(}\DataTypeTok{alpha=}\FloatTok{0.7}\NormalTok{) }\OperatorTok{+}
\StringTok{    }\KeywordTok{ylab}\NormalTok{(}\KeywordTok{paste0}\NormalTok{(}\StringTok{"Precision @ "}\NormalTok{, iou_threshold)) }\OperatorTok{+}
\StringTok{    }\KeywordTok{ylim}\NormalTok{(}\DecValTok{0}\NormalTok{,}\DecValTok{100}\NormalTok{) }\OperatorTok{+}
\StringTok{    }\KeywordTok{xlab}\NormalTok{(}\StringTok{"conv block"}\NormalTok{) }\OperatorTok{+}
\StringTok{    }\KeywordTok{ggtitle}\NormalTok{(}\KeywordTok{paste0}\NormalTok{(my_net, }\StringTok{" trained with "}\NormalTok{, lDS[[my_ds]])) }\OperatorTok{+}\StringTok{ }
\StringTok{    }\KeywordTok{theme}\NormalTok{(}\DataTypeTok{plot.title =} \KeywordTok{element_text}\NormalTok{(}\DataTypeTok{hjust =} \FloatTok{0.5}\NormalTok{))}
  \ControlFlowTok{if}\NormalTok{(my_iv }\OperatorTok{==}\StringTok{ 'back_distractor_target'}\NormalTok{)\{}
\NormalTok{    p =}\StringTok{ }\NormalTok{p }\OperatorTok{+}\StringTok{ }\KeywordTok{theme}\NormalTok{(}\DataTypeTok{legend.position=}\StringTok{"none"}\NormalTok{)}
    \CommentTok{# p = p + guides(colour=guide_legend(title = my_iv, nrow = 6, ncol = 10)) }
\NormalTok{  \} }\ControlFlowTok{else}\NormalTok{\{}
\NormalTok{    p =}\StringTok{ }\NormalTok{p }\OperatorTok{+}\StringTok{ }\KeywordTok{guides}\NormalTok{(}\DataTypeTok{colour=}\KeywordTok{guide_legend}\NormalTok{(}\DataTypeTok{title =}\NormalTok{ my_iv)) }
\NormalTok{  \}}
  \KeywordTok{return}\NormalTok{(p)}
\NormalTok{\}}
\end{Highlighting}
\end{Shaded}

\begin{Shaded}
\begin{Highlighting}[]
\NormalTok{get_colors =}\StringTok{ }\ControlFlowTok{function}\NormalTok{(my_file_name, my_iv, my_net, my_ds, my_row)\{}
\NormalTok{  df =}\StringTok{ }\KeywordTok{read.csv}\NormalTok{(my_file_name)}
  \CommentTok{# iou_threshold = 0.2}
  \CommentTok{# sd_threshold = 3}
  \ControlFlowTok{if}\NormalTok{(my_iv }\OperatorTok{==}\StringTok{ 'back_distractor_target'}\NormalTok{)\{}
\NormalTok{    df =}\StringTok{ }\KeywordTok{read.csv}\NormalTok{(my_file_name)}
\NormalTok{    df01 =}\StringTok{ }\NormalTok{df }\OperatorTok
\StringTok{      }\KeywordTok{select}\NormalTok{(}\KeywordTok{contains}\NormalTok{(}\StringTok{'color'}\NormalTok{)) }\CommentTok{#back_distractor_target}
\NormalTok{    identities =}\StringTok{ }\KeywordTok{c}\NormalTok{(}\StringTok{'back'}\NormalTok{, }\StringTok{'distractor'}\NormalTok{, }\StringTok{'target'}\NormalTok{)}
\NormalTok{    df02 =}\StringTok{ }\NormalTok{df01}
    \ControlFlowTok{for}\NormalTok{(iI }\ControlFlowTok{in}\NormalTok{ identities)\{}
\NormalTok{      df02 =}\StringTok{ }\KeywordTok{unite_}\NormalTok{(df02, iI, }\KeywordTok{colnames}\NormalTok{(}\KeywordTok{select}\NormalTok{(df02, }\KeywordTok{contains}\NormalTok{(iI))))}
\NormalTok{    \}}
\NormalTok{    df03 =}\StringTok{ }\NormalTok{df02}
    \ControlFlowTok{for}\NormalTok{(iC }\ControlFlowTok{in} \KeywordTok{names}\NormalTok{(lColor))\{}
\NormalTok{      df03 =}\StringTok{ }\KeywordTok{data.frame}\NormalTok{(}\KeywordTok{lapply}\NormalTok{(df03, }\ControlFlowTok{function}\NormalTok{(x)\{}
      \KeywordTok{str_replace}\NormalTok{(x, lColor[[iC]], iC)}
\NormalTok{      \}))}
\NormalTok{    \}}
\NormalTok{    df04 =}\StringTok{ }\KeywordTok{unite_}\NormalTok{(df03, my_iv, }\KeywordTok{colnames}\NormalTok{(df03))}
\NormalTok{    df =}\StringTok{ }\KeywordTok{cbind}\NormalTok{(df03, df04, df) }
\NormalTok{  \}}
\NormalTok{  df1 =}\StringTok{ }\NormalTok{df }\OperatorTok
\StringTok{    }\KeywordTok{select}\NormalTok{(}\KeywordTok{starts_with}\NormalTok{(}\StringTok{"iou"}\NormalTok{)) }\OperatorTok{>}\StringTok{ }\NormalTok{iou_threshold}
\NormalTok{  df2 =}\StringTok{ }\KeywordTok{data.frame}\NormalTok{(df1)}\OperatorTok{*}\DecValTok{100}
\NormalTok{  df3 =}\StringTok{ }\NormalTok{df2 }\OperatorTok
\StringTok{    }\KeywordTok{mutate}\NormalTok{(}\DataTypeTok{iv =}\NormalTok{ df[[my_iv]]) }\OperatorTok
\StringTok{    }\KeywordTok{group_by}\NormalTok{(iv) }\OperatorTok
\StringTok{    }\KeywordTok{summarise_all}\NormalTok{(}\KeywordTok{funs}\NormalTok{(}\StringTok{'mean'}\NormalTok{ =}\StringTok{ }\NormalTok{mean))}
  \CommentTok{# df4 = data.frame(t(df3))}
  \CommentTok{# df5 = df4[-1, ]}
  \CommentTok{# colnames(df5) = df4[1,]}
  \CommentTok{# df5$layer = 1:dim(df1)[2]}
  \CommentTok{# df6 = melt(df5, id='layer')}
\NormalTok{  df4 =}\StringTok{ }\KeywordTok{data.frame}\NormalTok{(}\KeywordTok{t}\NormalTok{(df3[, }\OperatorTok{-}\DecValTok{1}\NormalTok{]))}
  \CommentTok{# browser()}
  \KeywordTok{colnames}\NormalTok{(df4) =}\StringTok{ }\KeywordTok{sapply}\NormalTok{(df3[, }\DecValTok{1}\NormalTok{], as.character)}
\NormalTok{  df40 =}\StringTok{ }\NormalTok{df4}
\NormalTok{  color_mean =}\StringTok{ }\KeywordTok{rowMeans}\NormalTok{(df40)}
\NormalTok{  color_sd =}\StringTok{ }\KeywordTok{apply}\NormalTok{(df40, }\DecValTok{1}\NormalTok{, sd)}
\NormalTok{  cur_col =}\StringTok{ }\KeywordTok{data.frame}\NormalTok{(}\KeywordTok{matrix}\NormalTok{(}\DataTypeTok{ncol =} \DecValTok{1}\NormalTok{, }\DataTypeTok{nrow =} \DecValTok{2}\NormalTok{))}
  \KeywordTok{colnames}\NormalTok{(cur_col) =}\StringTok{ }\KeywordTok{paste0}\NormalTok{(iN, }\StringTok{" trained with "}\NormalTok{, lDS[[iD]])}
  \KeywordTok{rownames}\NormalTok{(cur_col) =}\StringTok{ }\KeywordTok{c}\NormalTok{(}\StringTok{"good_colors"}\NormalTok{, }\StringTok{"bad_colors"}\NormalTok{)}
\NormalTok{  df41 =}\StringTok{ }\KeywordTok{colSums}\NormalTok{(df40 }\OperatorTok{-}\StringTok{ }\NormalTok{color_mean }\OperatorTok{-}\StringTok{ }\NormalTok{sd_threshold }\OperatorTok{*}\StringTok{ }\NormalTok{color_sd }\OperatorTok{>}\StringTok{ }\DecValTok{0}\NormalTok{) }\OperatorTok{>}\StringTok{ }\DecValTok{0}
\NormalTok{  cur_col[}\StringTok{"good_colors"}\NormalTok{, ]=}\StringTok{ }\KeywordTok{paste}\NormalTok{(}\KeywordTok{names}\NormalTok{(}\KeywordTok{which}\NormalTok{(df41}\OperatorTok{==}\OtherTok{TRUE}\NormalTok{)), }\DataTypeTok{collapse =}\StringTok{' '}\NormalTok{)}
\NormalTok{  df41 =}\StringTok{ }\KeywordTok{colSums}\NormalTok{(color_mean }\OperatorTok{-}\StringTok{ }\NormalTok{sd_threshold }\OperatorTok{*}\StringTok{ }\NormalTok{color_sd }\OperatorTok{-}\StringTok{ }\NormalTok{df40 }\OperatorTok{>}\StringTok{ }\DecValTok{0}\NormalTok{) }\OperatorTok{>}\StringTok{ }\DecValTok{0}
\NormalTok{  cur_col[}\StringTok{"bad_colors"}\NormalTok{, ]=}\StringTok{ }\KeywordTok{paste}\NormalTok{(}\KeywordTok{names}\NormalTok{(}\KeywordTok{which}\NormalTok{(df41}\OperatorTok{==}\OtherTok{TRUE}\NormalTok{)), }\DataTypeTok{collapse =}\StringTok{' '}\NormalTok{)}
  \KeywordTok{return}\NormalTok{(cur_col)}


\NormalTok{\}}
\end{Highlighting}
\end{Shaded}

\begin{Shaded}
\begin{Highlighting}[]
\CommentTok{# iT = 'circles' # 'gratings' #}
\CommentTok{# iIv = 'back_distractor_target' #'num_item' #'back_distractor_target' #'target_angle' #}
\NormalTok{types =}\StringTok{ }\KeywordTok{c}\NormalTok{(}\StringTok{'circles'}\NormalTok{, }\StringTok{'gratings'}\NormalTok{)}
\NormalTok{lIV =}\StringTok{ }\KeywordTok{list}\NormalTok{(}\KeywordTok{c}\NormalTok{(}\StringTok{'num_item'}\NormalTok{),}
           \KeywordTok{c}\NormalTok{(}\StringTok{'num_item'}\NormalTok{, }\StringTok{'target_angle'}\NormalTok{))}
\KeywordTok{names}\NormalTok{(lIV) =}\StringTok{ }\NormalTok{types}

\NormalTok{nets =}\StringTok{ }\KeywordTok{c}\NormalTok{(}\StringTok{'vgg16'}\NormalTok{, }\StringTok{'res101'}\NormalTok{)}
\NormalTok{datasets =}\StringTok{ }\KeywordTok{c}\NormalTok{(}\StringTok{'pascal_voc'}\NormalTok{, }\StringTok{'vg'}\NormalTok{)}
\NormalTok{lDS =}\StringTok{ }\KeywordTok{c}\NormalTok{(}\StringTok{"pascal voc"}\NormalTok{, }\StringTok{"visual genome"}\NormalTok{)}
\KeywordTok{names}\NormalTok{(lDS) =}\StringTok{ }\NormalTok{datasets}


\ControlFlowTok{for}\NormalTok{ (iT }\ControlFlowTok{in}\NormalTok{ types)\{}
  \KeywordTok{print}\NormalTok{(iT)}
  \ControlFlowTok{for}\NormalTok{ (iIv }\ControlFlowTok{in}\NormalTok{ lIV[[iT]])\{}
    \KeywordTok{print}\NormalTok{(iIv)}
\NormalTok{    all_plots =}\StringTok{ }\KeywordTok{list}\NormalTok{()}
\NormalTok{    iP =}\StringTok{ }\DecValTok{1}
    \ControlFlowTok{for}\NormalTok{ (iN }\ControlFlowTok{in}\NormalTok{ nets)\{ }\CommentTok{# nets will be col}
      \ControlFlowTok{for}\NormalTok{ (iD }\ControlFlowTok{in}\NormalTok{ datasets)\{ }\CommentTok{# ds will be row}
\NormalTok{        file_name =}\StringTok{ }\KeywordTok{file.path}\NormalTok{(}\StringTok{'images'}\NormalTok{, }\KeywordTok{paste0}\NormalTok{(}\StringTok{'df_'}\NormalTok{, iT, }\StringTok{'_'}\NormalTok{, iN, }\StringTok{'_'}\NormalTok{, iD, }\StringTok{'.csv'}\NormalTok{))}
\NormalTok{        all_plots[[iP]] =}\StringTok{ }\KeywordTok{plot_results}\NormalTok{(file_name, iIv, iN, iD)}

\NormalTok{        iP =}\StringTok{ }\NormalTok{iP }\OperatorTok{+}\StringTok{ }\DecValTok{1}

        \CommentTok{# during testing}
        \CommentTok{# if(iP == 2)\{}
        \CommentTok{#   break}
        \CommentTok{# \}}
\NormalTok{      \}}
\NormalTok{    \}}
    \KeywordTok{ggarrange}\NormalTok{(}\DataTypeTok{plotlist =}\NormalTok{ all_plots, }\DataTypeTok{common.legend =} \OtherTok{TRUE}\NormalTok{, }\DataTypeTok{legend =} \StringTok{"right"}\NormalTok{)}
    \CommentTok{# browser()}
\NormalTok{    file_name =}\StringTok{ }\KeywordTok{paste0}\NormalTok{(iT, }\StringTok{'_'}\NormalTok{, iIv, }\StringTok{'.png'}\NormalTok{)}
    \KeywordTok{print}\NormalTok{(file_name)}
    \KeywordTok{ggsave}\NormalTok{(file_name)}
\NormalTok{  \}}
\NormalTok{\}}
\end{Highlighting}
\end{Shaded}

\begin{verbatim}
## [1] "circles"
## [1] "num_item"
## [1] "circles_num_item.png"
\end{verbatim}

\begin{verbatim}
## Saving 6.5 x 4.5 in image
\end{verbatim}

\begin{verbatim}
## [1] "gratings"
## [1] "num_item"
## [1] "gratings_num_item.png"
\end{verbatim}

\begin{verbatim}
## Saving 6.5 x 4.5 in image
\end{verbatim}

\begin{verbatim}
## [1] "target_angle"
## [1] "gratings_target_angle.png"
\end{verbatim}

\begin{verbatim}
## Saving 6.5 x 4.5 in image
\end{verbatim}

\begin{Shaded}
\begin{Highlighting}[]
\NormalTok{iT =}\StringTok{ 'circles'}
\NormalTok{iIv =}\StringTok{ 'back_distractor_target'}
\NormalTok{all_plots =}\StringTok{ }\KeywordTok{list}\NormalTok{()}
\NormalTok{iP =}\StringTok{ }\DecValTok{1}
\ControlFlowTok{for}\NormalTok{ (iN }\ControlFlowTok{in}\NormalTok{ nets)\{ }\CommentTok{# nets will be col}
  \ControlFlowTok{for}\NormalTok{ (iD }\ControlFlowTok{in}\NormalTok{ datasets)\{ }\CommentTok{# ds will be row}
\NormalTok{    file_name =}\StringTok{ }\KeywordTok{file.path}\NormalTok{(}\StringTok{'images'}\NormalTok{, }\KeywordTok{paste0}\NormalTok{(}\StringTok{'df_'}\NormalTok{, iT, }\StringTok{'_'}\NormalTok{, iN, }\StringTok{'_'}\NormalTok{, iD, }\StringTok{'.csv'}\NormalTok{))}
\NormalTok{    all_plots[[iP]] =}\StringTok{ }\KeywordTok{plot_results}\NormalTok{(file_name, iIv, iN, iD)}
      \ControlFlowTok{if}\NormalTok{(iP}\OperatorTok{==}\DecValTok{1}\NormalTok{)\{}
\NormalTok{        df_colors =}\StringTok{ }\KeywordTok{get_colors}\NormalTok{(file_name, iIv, iN, iD, iR)}
\NormalTok{      \} }\ControlFlowTok{else}\NormalTok{\{}
\NormalTok{        df_colors =}\StringTok{ }\KeywordTok{cbind}\NormalTok{(df_colors, }\KeywordTok{get_colors}\NormalTok{(file_name, iIv, iN, iD, iR))}
\NormalTok{      \}}
\NormalTok{      lDf_colors[[}\KeywordTok{paste}\NormalTok{(sd_threshold)]] =}\StringTok{ }\NormalTok{df_colors}
\NormalTok{      iP =}\StringTok{ }\NormalTok{iP }\OperatorTok{+}\StringTok{ }\DecValTok{1}

    \CommentTok{# during testing}
    \CommentTok{# if(iP == 2)\{}
    \CommentTok{#   break}
    \CommentTok{# \}}
      
\NormalTok{  \}}
\NormalTok{\}}
\KeywordTok{ggarrange}\NormalTok{(}\DataTypeTok{plotlist =}\NormalTok{ all_plots)}
\end{Highlighting}
\end{Shaded}

\includegraphics{Cindy_analysis_files/figure-latex/unnamed-chunk-6-1.pdf}

\begin{Shaded}
\begin{Highlighting}[]
\KeywordTok{ggsave}\NormalTok{(}\KeywordTok{paste0}\NormalTok{(iT, }\StringTok{'_'}\NormalTok{, iIv, }\StringTok{'.png'}\NormalTok{))}
\end{Highlighting}
\end{Shaded}

\begin{verbatim}
## Saving 6.5 x 4.5 in image
\end{verbatim}

\begin{Shaded}
\begin{Highlighting}[]
\CommentTok{# tg = all_plots}
\CommentTok{# ng = all_plots}
\CommentTok{# nc = all_plots}
\CommentTok{# cc = all_plots}
\CommentTok{# multiplot(plotlist = ng, cols=length(nets))}
\end{Highlighting}
\end{Shaded}

\begin{Shaded}
\begin{Highlighting}[]
\CommentTok{# df_gratings = df}
\CommentTok{# }
\CommentTok{# View(head(df_gratings))}
\CommentTok{# library(dplyr)}
\CommentTok{#                                           }
\CommentTok{# angles.df <- df_gratings[c("target_angle", "iou_module1_poolsize8", "iou_module2_poolsize8", }
\CommentTok{#                            "iou_module3_poolsize8", "iou_module4_poolsize8"}
\CommentTok{#                            , "iou_module5_poolsize8")]}
\CommentTok{# }
\CommentTok{# numitem.df <-  df_gratings[c("num_item", "iou_module1_poolsize8", "iou_module2_poolsize8", }
\CommentTok{#                              "iou_module3_poolsize8", "iou_module4_poolsize8"}
\CommentTok{#                              , "iou_module5_poolsize8")]}
\CommentTok{# }
\CommentTok{# }
\CommentTok{# for(colnum in 2:ncol(numitem.df))\{}
\CommentTok{#   for(rownum in 1:nrow(numitem.df))\{}
\CommentTok{#     if(numitem.df[rownum, colnum] >= .2)\{}
\CommentTok{#       numitem.df[rownum, colnum] <- 1}
\CommentTok{#     \} else if(numitem.df[rownum, colnum] < .2)\{}
\CommentTok{#       numitem.df[rownum, colnum] <- 0 }
\CommentTok{#     \}}
\CommentTok{#   \}}
\CommentTok{# \}}
\CommentTok{# }
\CommentTok{# for(colnum in 2:ncol(angles.df))\{}
\CommentTok{#   for(rownum in 1:nrow(angles.df))\{}
\CommentTok{#     if(angles.df[rownum, colnum] >= .2)\{}
\CommentTok{#       angles.df[rownum, colnum] <- 1}
\CommentTok{#     \} else if(angles.df[rownum, colnum] < .2)\{}
\CommentTok{#       angles.df[rownum, colnum] <- 0 }
\CommentTok{#     \}}
\CommentTok{#   \}}
\CommentTok{# \}}
\CommentTok{# }
\CommentTok{# }
\CommentTok{# grouped.angles <- angles.df %>% group_by(target_angle)}
\CommentTok{# }
\CommentTok{# sumr.angles <- data.frame(summarize(grouped.angles, sum.layer1 = sum(iou_module1_poolsize8), }
\CommentTok{#                                                    sum.layer2 = sum(iou_module2_poolsize8),}
\CommentTok{#                                                    sum.layer3 = sum(iou_module3_poolsize8),}
\CommentTok{#                                                    sum.layer4 = sum(iou_module4_poolsize8),}
\CommentTok{#                                                    sum.layer5 = sum(iou_module5_poolsize8)))}
\CommentTok{# }
\CommentTok{# grouped.numitems <- numitem.df %>% group_by(num_item)}
\CommentTok{# sumr.numitems <- data.frame(summarize(grouped.numitems, sum.layer1 = sum(iou_module1_poolsize8), }
\CommentTok{#                                                    sum.layer2 = sum(iou_module2_poolsize8),}
\CommentTok{#                                                    sum.layer3 = sum(iou_module3_poolsize8),}
\CommentTok{#                                                    sum.layer4 = sum(iou_module4_poolsize8),}
\CommentTok{#                                                    sum.layer5 = sum(iou_module5_poolsize8)))}
\CommentTok{# row.names(sumr.angles) <- sumr.angles[,1]}
\CommentTok{# row.names(sumr.numitems) <- sumr.numitems[,1]}
\CommentTok{# }
\CommentTok{# sumr.numitems <- sumr.numitems[order(sumr.numitems$num_item),]}
\CommentTok{# sumr.angles <- sumr.angles[order(sumr.angles$target_angle),]}
\CommentTok{# angles.m <- data.matrix(sumr.angles)}
\CommentTok{# numitem.m <- data.matrix((sumr.numitems))}
\CommentTok{# }
\CommentTok{# }
\CommentTok{# heatmap.numitems <- heatmap(numitem.m[,2:6], Rowv=NA, Colv=NA, }
\CommentTok{#                        col = heat.colors(256), }
\CommentTok{#                        scale="column", margins=c(10,5))}
\CommentTok{# }
\CommentTok{# heatmap.angles <- heatmap(angles.m[,2:6], Rowv = NA, Colv = NA, }
\CommentTok{#                           col = heat.colors(256), }
\CommentTok{#                           scale = "column", margins = c(8, 10))}
\end{Highlighting}
\end{Shaded}


\end{document}
